\documentclass{beamer}
\mode<presentation>
\usepackage{amsmath}
\usepackage{amssymb}
%\usepackage{advdate}
\usepackage{adjustbox}
\usepackage{subcaption}
\usepackage{enumitem}
\usepackage{multicol}
\usepackage{mathtools}
\usepackage{listings}
\usepackage{url}
\def\UrlBreaks{\do\/\do-}
\usetheme{Boadilla}
\usecolortheme{lily}
\setbeamertemplate{footline}
{
  \leavevmode%
  \hbox{%
  \begin{beamercolorbox}[wd=\paperwidth,ht=2.25ex,dp=1ex,right]{author in head/foot}%
    \insertframenumber{} / \inserttotalframenumber\hspace*{2ex} 
  \end{beamercolorbox}}%
  \vskip0pt%
}
\setbeamertemplate{navigation symbols}{}

\providecommand{\nCr}[2]{\,^{#1}C_{#2}} % nCr
\providecommand{\nPr}[2]{\,^{#1}P_{#2}} % nPr
\providecommand{\mbf}{\mathbf}
\providecommand{\pr}[1]{\ensuremath{\Pr\left(#1\right)}}
\providecommand{\qfunc}[1]{\ensuremath{Q\left(#1\right)}}
\providecommand{\sbrak}[1]{\ensuremath{{}\left[#1\right]}}
\providecommand{\lsbrak}[1]{\ensuremath{{}\left[#1\right.}}
\providecommand{\rsbrak}[1]{\ensuremath{{}\left.#1\right]}}
\providecommand{\brak}[1]{\ensuremath{\left(#1\right)}}
\providecommand{\lbrak}[1]{\ensuremath{\left(#1\right.}}
\providecommand{\rbrak}[1]{\ensuremath{\left.#1\right)}}
\providecommand{\cbrak}[1]{\ensuremath{\left\{#1\right\}}}
\providecommand{\lcbrak}[1]{\ensuremath{\left\{#1\right.}}
\providecommand{\rcbrak}[1]{\ensuremath{\left.#1\right\}}}
\theoremstyle{remark}
\newtheorem{rem}{Remark}
\newcommand{\sgn}{\mathop{\mathrm{sgn}}}
\providecommand{\abs}[1]{\left\vert#1\right\vert}
\providecommand{\res}[1]{\Res\displaylimits_{#1}} 
\providecommand{\norm}[1]{\lVert#1\rVert}
\providecommand{\mtx}[1]{\mathbf{#1}}
\providecommand{\mean}[1]{E\left[ #1 \right]}
\providecommand{\fourier}{\overset{\mathcal{F}}{ \rightleftharpoons}}
%\providecommand{\hilbert}{\overset{\mathcal{H}}{ \rightleftharpoons}}
\providecommand{\system}[1]{\overset{\mathcal{#1}}{ \longleftrightarrow}}
%\providecommand{\system}{\overset{\mathcal{H}}{ \longleftrightarrow}}
	%\newcommand{\solution}[2]{\textbf{Solution:}{#1}}
%\newcommand{\solution}{\noindent \textbf{Solution: }}
\providecommand{\dec}[2]{\ensuremath{\overset{#1}{\underset{#2}{\gtrless}}}}
\newcommand{\myvec}[1]{\ensuremath{\begin{pmatrix}#1\end{pmatrix}}}
\let\vec\mathbf

\lstset{
%language=C,
frame=single, 
breaklines=true,
columns=fullflexible
}

\numberwithin{equation}{section}

\title{Signal Processing in High School}
\author{G. V. V. Sharma \\ Dept. of Electrical Engg.,\\IIT Hyderabad.}

\date{\today} 
\begin{document}

\begin{frame}
\titlepage
\end{frame}

\section*{Outline}
\begin{frame}
\tableofcontents
\end{frame}
\section{Two Dice}
\begin{frame}
\frametitle{NCERT}
Two dice, one blue and one grey, are thrown at the same time.   The event defined by the sum of the two numbers appearing on the top of the dice can have 11 possible outcomes 2, 3, 4, 5, 6, 6, 8, 9, 10, 11 and 12.  A student argues that each of these outcomes has a probability $\frac{1}{11}$.  Do you agree with this argument?  Justify your answer.
\end{frame}

\begin{frame}
\frametitle
	{Uniform Distribution: Rectangular Function}
		Let $X_i \in \cbrak{1,2,3,4,5,6}, i = 1,2,$ be the random variables representing the outcome for each die.  Assuming the dice to be fair, the probability mass function (pmf) is expressed as 
\begin{align}
\label{eq:dice_pmf_xi}
p_{X_i}(n) = \pr{X_i = n} = 
\begin{cases}
\frac{1}{6} & 1 \le n \le 6
\\
0 & otherwise
\end{cases}
\end{align}
\end{frame}
\begin{frame}
\frametitle{Sum of Random Variables: Convolution}
\begin{enumerate}[label=\thesection.\arabic*.,ref=\thesection.\theenumi]
	\item The desired outcome is
\begin{align}
\label{eq:dice_xdef}
X &= X_1 + X_2,
\\
\implies X &\in \cbrak{1,2,\dots,12}
\end{align}
%
The objective is to show that
\begin{align}
p_X(n) \ne \frac{1}{11}
\label{eq:dice_wrong}
\end{align}
\item {\em Convolution: }
From \eqref{eq:dice_xdef},
\begin{align}
p_X(n) &= \pr{X_1 + X_2 = n} = \pr{X_1  = n -X_2}
\\
&= \sum_{k}^{}\pr{X_1  = n -k | X_2 = k}p_{X_2}(k)
\label{eq:dice_x_sum}
\end{align}
after unconditioning.  $\because X_1$ and $X_2$ are independent,
\begin{align}
&\pr{X_1  = n -k | X_2 = k} 
	= \pr{X_1  = n -k} = p_{X_1}(n-k)
%\label{eq:dice_x1_indep}
\nonumber \\
\implies
%\end{align}
%From \eqref{eq:dice_x_sum} and \eqref{eq:dice_x1_indep},
%\begin{align}
	&p_X(n) = \sum_{k}^{}p_{X_1}(n-k)p_{X_2}(k) \triangleq p_{X_1}(n)*p_{X_2}(n)
\label{eq:dice_x_conv}
\end{align}
where $*$ denotes the convolution operation. 
\end{enumerate}
\end{frame}
\begin{frame}
\frametitle{Convolution Cont.}
\begin{align}
p_X(n) = \frac{1}{6}\sum_{k=1}^{6}p_{X_1}(n-k)= \frac{1}{6}\sum_{k=n-6}^{n-1}p_{X_1}(k)
\label{eq:dice_x_conv_x1}
\end{align}
\begin{align}
\because p_{X_1}(k) &= 0, \quad k \le 1, k \ge 6.
\end{align}
From \eqref{eq:dice_x_conv_x1},
%
\begin{align}
p_X(n) &= 
\begin{cases}
0 & n < 1
\\
\frac{1}{6}\sum_{k=1}^{n-1}p_{X_1}(k) &  1 \le n-1 \le  6
\\
\frac{1}{6}\sum_{k=n-6}^{6}p_{X_1}(k) & 1 < n-6 \le 6
\\
0 & n > 12
\end{cases}
\label{eq:dice_x_conv_cond}
\end{align}
\end{frame}
\begin{frame}
\frametitle{Triangular Distribution: Student is wrong}
\begin{align}
p_X(n) &= 
\begin{cases}
0 & n < 1
\\
\frac{n-1}{36} &  2 \le n \le  7
\\
\frac{13-n}{36} & 7 < n \le 12
\\
0 & n > 12
\end{cases}
\label{eq:dice_x_conv_final}
\end{align}
\end{frame}
\begin{frame}
\frametitle{Z Transform}
The $Z$-transform of $p_X(n)$ is defined as 
%\cite{proakis_dsp}
\begin{align}
P_X(z) = \sum_{n = -\infty}^{\infty}p_X(n)z^{-n}, \quad z \in \mathbb{C}
\label{eq:dice_xz}
\end{align}
%
From \eqref{eq:dice_pmf_xi} and \eqref{eq:dice_xz}, 
\begin{align}
P_{X_1}(z) =P_{X_2}(z) &= \frac{1}{6}\sum_{n = 1}^{6}z^{-n}
\\
&=\frac{z^{-1}\brak{1-z^{-6}}}{6\brak{1-z^{-1}}}, \quad \abs{z} > 1
\label{eq:dice_xiz}
\end{align}
upon summing up the geometric progression.  
\end{frame}
\begin{frame}
\frametitle{Convolution vs Multiplication}
\begin{align}
p_X(n) &= p_{X_1}(n)*p_{X_2}(n)
\implies
P_X(z) = P_{X_1}(z)P_{X_2}(z)
\label{eq:dice_xzprod_def}
\end{align}
Thus,
\begin{align}
P_X(z) = \cbrak{\frac{z^{-1}\brak{1-z^{-6}}}{6\brak{1-z^{-1}}}}^2
= \frac{1}{36}\frac{z^{-2}\brak{1-2z^{-6}+z^{-12}}}{\brak{1-z^{-1}}^2}
\label{eq:dice_xzprod}
\end{align}
\end{frame}
\begin{frame}
\frametitle{Inverse Z transform}
		The $Z$ transform of $u(n)$ is 
\begin{align}
u(n)&\system{Z} 
 \frac{1}{1-z^{-1}},\,\abs{z} > 1
\end{align}
and
\begin{align}
\label{eq:dice-ramp}
nu(n)&\system{Z} \frac{z^{-1}}{\brak{1-z^{-1}}^2},\,
\abs{z} > 1
\end{align}
\begin{multline}
\frac{1}{36}\sbrak{\brak{n-1}u(n-1) - 2 \brak{n-7}u(n-7)
	+\brak{n-13}u(n-13)}
	\\
\system{Z}
\frac{1}{36}\frac{z^{-2}\brak{1-2z^{-6}+z^{-12}}}{\brak{1-z^{-1}}^2}
\label{eq:dice_xz_closed}
\end{multline}
\end{frame}
\section{Pingala Series}
\begin{frame}
\frametitle{Problem Statement}
Let
\begin{align}
	\label{eq:pingala/10-orig-diff-a}
	a_n &= \frac{\alpha^{n}-\beta^{n}}{\alpha - \beta}, \quad n \ge 1
	\\
	b_n &= a_{n-1} + a_{n+1}, \quad n \ge 2, \quad b_1 =1
	\label{eq:pingala/10-orig-diff}
\end{align}
where $\alpha$ and $\beta$ ($\alpha > \beta$) are the roots of the
\begin{align}
z^2 - z - 1 = 0
\end{align}
%
\end{frame}
\begin{frame}
\frametitle{JEE 2019}
Which of the following options is correct?
\begin{enumerate}[label=\thesection.\arabic*,ref=\thesection.\theenumi]
\item 
	\label{itm:ping-1}
\begin{align}
	\label{eq:ping-1}
	\sum_{k=1}^{n}a_k = a_{n+2}-1, \quad n \ge 1
\end{align}
 \item 
	\label{itm:ping-2}
\begin{align}
	\label{eq:ping-2}
	\sum_{k=1}^{\infty}\frac{a_k}{10^k} =\frac{10}{89}
\end{align}
 \item 
	\label{itm:ping-3}
\begin{align}
	\label{eq:ping-3}
	b_n =\alpha^n + \beta^n, \quad n \ge 1
\end{align}
 \item 
	\label{itm:ping-4}
\begin{align}
	\label{eq:ping-4}
	\sum_{k=1}^{\infty}\frac{b_k}{10^k} =\frac{8}{89}
\end{align}
\end{enumerate}
\end{frame}
\begin{frame}
\frametitle{Difference Equation}
The {\em Pingala} series is generated using the difference equation 
\begin{align}
	x(n+2) = x\brak{n+1} + x\brak{n},  \quad x(0) = x(1) = 1, n \ge 0
	\label{eq:pingala/10-pingala}
\end{align}
\end{frame}
\begin{frame}
\frametitle{One Sided Z transform}
The {\em one sided} $Z$-transform of $x(n)$ is defined as 
\begin{align}
	X^{+}(z) = \sum_{n = 0}^{\infty}x(n)z^{-n}, \quad z \in \mathbb{C}
\label{eq:pingala/one-Z}
\end{align}
Taking the one-sided $Z$-transform on both sides of \eqref{eq:pingala/10-pingala},
\begin{align}
	\mathcal{Z}^+\sbrak{x(n + 2)} &= \mathcal{Z}^+\sbrak{x(n + 1)} + \mathcal{Z}^+\sbrak{x(n)} \\
\implies     z^2X^+(z) - z^2x(0) - zx(1) &= zX^+(z) - zx(0) + zX^+(z) \\
 \implies   \brak{z^2 - z - 1}X^+(z) &= z^2 \\
  \implies    X^+(z) = \frac{1}{1 - z^{-1} - z^{-2}} 
    &= \frac{1}{\brak{1 - \alpha z^{-1}}\brak{1 - \beta z^{-1}}}, \quad |z| > \alpha
    \label{eq:pingala/X-z}
\end{align}
\end{frame}
\begin{frame}
	\frametitle{Inverse}
Expanding $X^+(z)$ in \eqref{eq:pingala/X-z} using partial fractions, we get
\begin{align}
    X^+(z) &= \frac{1}{\brak{\alpha - \beta}}\sbrak{\frac{z}{1 - \alpha z^{-1}} - \frac{z}{1 - \beta z^{-1}}} \\
	\implies    x(n) &= \frac{\alpha^{n + 1} - \beta^{n + 1}}{\alpha - \beta}u(n) 
	\\
	&= a_{n + 1}
    \label{eq:pingala/x-n-def}
\end{align}
upon comparing with
	\eqref{eq:pingala/10-orig-diff-a}.
\end{frame}
\begin{frame}
	\frametitle{Linear Time Invariant System}
	Let 
\begin{align}
	y(n) = x\brak{n-1} + x\brak{n+1},  \quad n \ge 0
	\label{eq:pingala/10-orig-diff-rev}
\end{align}
	Taking the one-sided $Z$-transform on both sides of \eqref{eq:pingala/10-orig-diff-rev},
\begin{align}
	\mathcal{Z}^+\sbrak{y(n)} &= \mathcal{Z}^+\sbrak{x(n + 1)} + \mathcal{Z}^+\sbrak{x(n - 1)} \\
	Y^+(z) &= zX^+(z) - zx(0) + z^{-1}X^+(z) + zx(-1) \\
&= \frac{z + z^{-1}}{1 - z^{-1} - z^{-2}} - z 
= \frac{1 + 2z^{-1}}{1 - z^{-1} - z^{-2}}, \quad |z| > \alpha
\end{align}
\end{frame}
\begin{frame}
	\frametitle{Power of the Z transform}
\begin{align}
	\alpha^n + \beta^n, \quad n \ge 1
    \label{eq:pingala/yn-exp}
\end{align}
can be expressed as 
\begin{align}
	w(n) = \brak{\alpha^{n+1} + \beta^{n+1}}u(n)
\end{align}
Therefore,
\begin{align}
    W(z) = Y(z) = \frac{1 + 2z^{-1}}{1 - z^{-1} - z^{-2}}
\end{align}
\end{frame}
\begin{frame}
\frametitle{Power of the Z transform Cont.}
\begin{align}
    \sum_{k=1}^{\infty}\frac{a_k}{10^k} &= \frac{1}{10}\sum_{k = 0}^{\infty}\frac{a_{k+1}}{10^k} 
                                        = \frac{1}{10}\sum_{k = 0}^{\infty}\frac{x(k)}{10^k} \\
                                        &= \frac{1}{10}X^+(10) 
                                        = \frac{1}{10}\times\frac{100}{89} = \frac{10}{89}
\end{align}
Thus,
\eqref{eq:ping-2} is correct.
\begin{align}
    \sum_{k=1}^{\infty}\frac{b_k}{10^k} &= \frac{1}{10}\sum_{k = 0}^{\infty}\frac{b_{k+1}}{10^k} 
                                        = \frac{1}{10}\sum_{k = 0}^{\infty}\frac{y(k)}{10^k} \\
                                        &= \frac{1}{10}Y^+(z) 
                                        = \frac{1}{10}\times\frac{120}{89} = \frac{12}{89}
\end{align}
Thus,
\eqref{eq:ping-4} is incorrect.
\end{frame}
\begin{frame}
\frametitle{Convolution}
\begin{align}
    \sum_{k=1}^{n}a_k &= \sum_{k=0}^{n-1}x(k) 
                      = \sum_{k = -\infty}^{n - 1}x(k) \\
                      &= \sum_{k = -\infty}^{\infty}x(k)u(n - 1 - k) 
                      = x(n)*u(n - 1)
\end{align}
\begin{align}
a_{n+2}-1, \quad n \ge 1
\end{align}
can be expressed as 
\begin{align}
	\sbrak{x\brak{n+1}-1}u\brak{n-1}
\end{align}
\end{frame}
\begin{frame}
\frametitle{Convolution Cont.}
The Z transform of the above signal can be expressed as
\begin{align}
	\sum_{n = 1}^{\infty}x(n + 1) z^{-n} -\frac{z^{-1}}{1-z^{-1}}
	&=\sum_{n = 2}^{\infty}x(n) z^{-n+1} -\frac{z^{-1}}{1-z^{-1}}
	\\
	&=z\sbrak{X^{+}(z) - x(0) -x(1)z^{-1}} -\frac{z^{-1}}{1-z^{-1}}
	\\
	&= \frac{z}{1 - z^{-1} - z^{-2}} - z - 1 - \frac{z^{-1}}{1 - z^{-1}} \\
	&= \frac{z}{1 - z^{-1} - z^{-2}} -  \frac{z}{1 - z^{-1}} \\
	&=\frac{z^{-1}}{\brak{1 - z^{-1} - z^{-2}}\brak{1 - z^{-1}}} 
\end{align}
From \eqref{eq:pingala/x-n-def}, we get
\begin{align}
    \sum_{k = 1}^{n}a_k = a_{n+2} - 1
\end{align}
\end{frame}
%\section{FFT}
% \subsection{Definitions}
\begin{enumerate}[label=\arabic*.,ref=\thesection.\theenumi]
\numberwithin{equation}{enumi}
    \item The DFT of $x(n)$ is given by
    \begin{align}
        X(k) \triangleq \sum_{n=0}^{N-1} x(n) e^{-j 2 \pi k n / N}, \quad k=0,1, \ldots, N-1
	\label{eq:app-dft-def}
    \end{align}
\item Let 
	\begin{align}
W_{N} = e^{-j2\pi/N} 
	\end{align}
		Then the $N$-point {\em DFT matrix} is defined as 
	\begin{align}
		\vec{F}_{N} = \sbrak{W_{N}^{mn}}, \quad 0 \le m,n \le N-1 
	\end{align}
	where $W_{N}^{mn}$ are the elements of $\vec{F}_{N}$.
 \end{enumerate}
\section{FFT}
% \subsection{Definitions}
\begin{enumerate}[label=\arabic*.,ref=\thesection.\theenumi]
\numberwithin{equation}{enumi}
\item Let 
	\begin{align}
		\vec{I}_4 = \myvec{\vec{e}_4^{1} &\vec{e}_4^{2} &\vec{e}_4^{3} &\vec{e}_4^{4} }
	\end{align}
		be the $4\times 4$ identity matrix.  Then the 4 point {\em DFT permutation matrix} is defined as 
	\begin{align}
		\vec{P}_4 = \myvec{\vec{e}_4^{1} &\vec{e}_4^{3} &\vec{e}_4^{2} &\vec{e}_4^{4} }
	\end{align}
\item The 4 point {\em DFT diagonal matrix} is defined as 
	\begin{align}
		\vec{D}_4 = diag\myvec{W_{8}^{0} & W_{8}^{1} & W_{8}^{2} & W_{8}^{3}}
	\end{align}
\item Show that 
\begin{equation}
    W_{N}^{2}=W_{N/2}
\end{equation}

%    \item Find $\vec{P}_6$.
%    \item Find $\vec{D}_3$.
    \item Show that 
\begin{equation}
	\vec{F}_{4}=
\begin{bmatrix}
	\vec{I}_{2} & \vec{D}_{2} \\
\vec{I}_{2} & -\vec{D}_{2}
\end{bmatrix}
\begin{bmatrix}
\vec{F}_{2} & 0 \\
0 & \vec{F}_{2}
\end{bmatrix}
\vec{P}_{4}
\end{equation}
\item Show that 
\begin{equation}
\vec{F}_{N}=
\begin{bmatrix}
\vec{I}_{N/2} & \vec{D}_{N/2} \\
\vec{I}_{N/2} & -\vec{D}_{N/2}
\end{bmatrix}
\begin{bmatrix}
\vec{F}_{N/2} & 0 \\
0 & \vec{F}_{N/2}
\end{bmatrix}
\vec{P}_{N}
\end{equation}

\item Find 
    \begin{align}
	     \vec{P}_4 \vec{x}
    \end{align}
\item Show that 
    \begin{align}
	    \vec{X} = \vec{F}_N \vec{x}
	    \label{eq:dft-mat-def}
    \end{align}
		where $\vec{x}, \vec{X}$ are the vector representations of $x(n), X(k)$ respectively.
\item Derive the following Step-by-step visualisation  of
8-point FFTs into 4-point FFTs and so on
\begin{equation}
\begin{bmatrix}
X(0) \\ 
X(1) \\ 
X(2) \\ 
X(3)
\end{bmatrix}
=
\begin{bmatrix}
X_{1}(0) \\ 
X_{1}(1)\\ 
X_{1}(2)\\
X_{1}(3)\\
\end{bmatrix}
+
\begin{bmatrix}
W^{0}_{8} & 0 & 0 & 0\\
0 & W^{1}_{8} & 0 & 0\\
0 & 0 & W^{2}_{8} & 0\\
0 & 0 & 0 & W^{3}_{8}
\end{bmatrix}
\begin{bmatrix}
X_{2}(0) \\ 
X_{2}(1) \\ 
X_{2}(2) \\
X_{2}(3)
\end{bmatrix}
\end{equation}

\begin{equation}
\begin{bmatrix}
X(4) \\ 
X(5) \\ 
X(6) \\ 
X(7)
\end{bmatrix}
=
\begin{bmatrix}
X_{1}(0) \\ 
X_{1}(1)\\ 
X_{1}(2)\\
X_{1}(3)\\
\end{bmatrix}
-
\begin{bmatrix}
W^{0}_{8} & 0 & 0 & 0\\
0 & W^{1}_{8} & 0 & 0\\
0 & 0 & W^{2}_{8} & 0\\
0 & 0 & 0 & W^{3}_{8}
\end{bmatrix}
\begin{bmatrix}
X_{2}(0) \\ 
X_{2}(1) \\ 
X_{2}(2) \\
X_{2}(3)
\end{bmatrix}
\end{equation}

4-point FFTs into 2-point FFTs
\begin{equation}
\begin{bmatrix}
X_{1}(0) \\ 
X_{1}(1)\\ 
\end{bmatrix}
=
\begin{bmatrix}
X_{3}(0) \\ 
X_{3}(1)\\ 
\end{bmatrix}
+
\begin{bmatrix}
W^{0}_{4} & 0\\
0 & W^{1}_{4}
\end{bmatrix}
\begin{bmatrix}
X_{4}(0) \\ 
X_{4}(1) \\ 
\end{bmatrix}
\end{equation}

\begin{equation}
\begin{bmatrix}
X_{1}(2) \\ 
X_{1}(3)\\ 
\end{bmatrix}
=
\begin{bmatrix}
X_{3}(0) \\ 
X_{3}(1)\\ 
\end{bmatrix}
-
\begin{bmatrix}
W^{0}_{4} & 0\\
0 & W^{1}_{4}
\end{bmatrix}
\begin{bmatrix}
X_{4}(0) \\ 
X_{4}(1) \\ 
\end{bmatrix}
\end{equation}

\begin{equation}
\begin{bmatrix}
X_{2}(0) \\ 
X_{2}(1)\\ 
\end{bmatrix}
=
\begin{bmatrix}
X_{5}(0) \\ 
X_{5}(1)\\ 
\end{bmatrix}
+
\begin{bmatrix}
W^{0}_{4} & 0\\
0 & W^{1}_{4}
\end{bmatrix}
\begin{bmatrix}
X_{6}(0) \\ 
X_{6}(1) \\ 
\end{bmatrix}
\end{equation}

\begin{equation}
\begin{bmatrix}
X_{2}(2) \\ 
X_{2}(3)\\ 
\end{bmatrix}
=
\begin{bmatrix}
X_{5}(0) \\ 
X_{5}(1)\\ 
\end{bmatrix}
-
\begin{bmatrix}
W^{0}_{4} & 0\\
0 & W^{1}_{4}
\end{bmatrix}
\begin{bmatrix}
X_{6}(0) \\ 
X_{6}(1) \\ 
\end{bmatrix}
\end{equation}

\begin{equation}
P_{8}
\begin{bmatrix}
x(0) \\ 
x(1) \\ 
x(2) \\ 
x(3) \\ 
x(4) \\ 
x(5) \\
x(6) \\
x(7)
\end{bmatrix}
 = 
\begin{bmatrix}
x(0) \\ 
x(2) \\ 
x(4) \\ 
x(6) \\
x(1) \\ 
x(3) \\ 
x(5) \\
x(7)
\end{bmatrix}
\end{equation}

\begin{equation}
P_{4}
\begin{bmatrix}
x(0) \\ 
x(2) \\ 
x(4) \\ 
x(6) \\
\end{bmatrix}
 = 
\begin{bmatrix}
x(0) \\ 
x(4) \\ 
x(2) \\
x(6)
\end{bmatrix}
\end{equation}

\begin{equation}
P_{4}
\begin{bmatrix}
x(1) \\ 
x(3) \\ 
x(5) \\
x(7)
\end{bmatrix}
 = 
\begin{bmatrix}
x(1) \\ 
x(5) \\ 
x(3) \\ 
x(7) \\
\end{bmatrix}
\end{equation}

Therefore,
\begin{equation}
\begin{bmatrix}
X_{3}(0) \\ 
X_{3}(1)\\ 
\end{bmatrix}
= F_{2}
\begin{bmatrix}
x(0) \\ 
x(4) \\ 
\end{bmatrix}
\end{equation}

\begin{equation}
\begin{bmatrix}
X_{4}(0) \\ 
X_{4}(1)\\ 
\end{bmatrix}
= F_{2}
\begin{bmatrix}
x(2) \\ 
x(6) \\ 
\end{bmatrix}
\end{equation}

\begin{equation}
\begin{bmatrix}
X_{5}(0) \\ 
X_{5}(1)\\ 
\end{bmatrix}
= F_{2}
\begin{bmatrix}
x(1) \\ 
x(5) \\ 
\end{bmatrix}
\end{equation}

\begin{equation}
\begin{bmatrix}
X_{6}(0) \\ 
X_{6}(1)\\ 
\end{bmatrix}
= F_{2}
\begin{bmatrix}
x(3) \\ 
x(7) \\ 
\end{bmatrix}
\end{equation}

\item For 

    \begin{align}
	    \vec{x} = \myvec{1\\2\\3\\4\\2\\1}
        \label{eq:equation1}
    \end{align}
    compte the DFT  
		using 
	    \eqref{eq:dft-mat-def}
    \item Repeat the above exercise using the FFT
	    after zero padding $\vec{x}$.
%	    \eqref{eq:fft-mat-def}
\item Write a C program to compute the 8-point FFT. 
 \end{enumerate}

\section{Power Spectral Density(PSD)}
% \subsection{Definitions}
\begin{enumerate}[label=\arabic*.,ref=\thesection.\theenumi]
\numberwithin{equation}{enumi}
    \item Power spectral density (PSD) is a measure of the power distribution over frequency components of a signal.The PSD of \eqref{eq:app-dft-def}is given by\\
   \begin{align}
X(k)=\lvert X(k) \rvert^2 
\label{eq:app-psd-def}
    \end{align}

 \end{enumerate}
%\chapter{Pingala Series}
%\iffalse
\documentclass[journal,12pt,twocolumn]{IEEEtran}
%
\usepackage{setspace}
\usepackage{gensymb}
\usepackage{xcolor}
\usepackage{caption}
%\usepackage{subcaption}
%\doublespacing
\singlespacing

%\usepackage{graphicx}
%\usepackage{amssymb}
%\usepackage{relsize}
\usepackage[cmex10]{amsmath}
\usepackage{mathtools}
%\usepackage{amsthm}
%\interdisplaylinepenalty=2500
%\savesymbol{iint}
%\usepackage{txfonts}
%\restoresymbol{TXF}{iint}
%\usepackage{wasysym}
\usepackage{hyperref}
\usepackage{amsthm}
\usepackage{mathrsfs}
\usepackage{txfonts}
\usepackage{stfloats}
\usepackage{cite}
\usepackage{cases}
\usepackage{subfig}
%\usepackage{xtab}
\usepackage{longtable}
\usepackage{multirow}
%\usepackage{algorithm}
%\usepackage{algpseudocode}
%\usepackage{enumerate}
\usepackage{enumitem}
\usepackage{mathtools}
%\usepackage{iithtlc}
%\usepackage[framemethod=tikz]{mdframed}
\usepackage{listings}
\let\vec\mathbf


%\usepackage{stmaryrd}


%\usepackage{wasysym}
%\newcounter{MYtempeqncnt}
\DeclareMathOperator*{\Res}{Res}
%\renewcommand{\baselinestretch}{2}
\renewcommand\thesection{\arabic{section}}
\renewcommand\thesubsection{\thesection.\arabic{subsection}}
\renewcommand\thesubsubsection{\thesubsection.\arabic{subsubsection}}

\renewcommand\thesectiondis{\arabic{section}}
\renewcommand\thesubsectiondis{\thesectiondis.\arabic{subsection}}
\renewcommand\thesubsubsectiondis{\thesubsectiondis.\arabic{subsubsection}}

%\renewcommand{\labelenumi}{\textbf{\theenumi}}
%\renewcommand{\theenumi}{P.\arabic{enumi}}

% correct bad hyphenation here
\hyphenation{op-tical net-works semi-conduc-tor}

\lstset{
language=Python,
frame=single, 
breaklines=true,
columns=fullflexible
}



\begin{document}
%

\theoremstyle{definition}
\newtheorem{theorem}{Theorem}[section]
\newtheorem{problem}{Problem}
\newtheorem{proposition}{Proposition}[section]
\newtheorem{lemma}{Lemma}[section]
\newtheorem{corollary}[theorem]{Corollary}
\newtheorem{example}{Example}[section]
\newtheorem{definition}{Definition}[section]
%\newtheorem{algorithm}{Algorithm}[section]
%\newtheorem{cor}{Corollary}
\newcommand{\BEQA}{\begin{eqnarray}}
\newcommand{\EEQA}{\end{eqnarray}}
\newcommand{\define}{\stackrel{\triangle}{=}}
\newcommand{\myvec}[1]{\ensuremath{\begin{pmatrix}#1\end{pmatrix}}}
\newcommand{\mydet}[1]{\ensuremath{\begin{vmatrix}#1\end{vmatrix}}}

\bibliographystyle{IEEEtran}
%\bibliographystyle{ieeetr}

\providecommand{\nCr}[2]{\,^{#1}C_{#2}} % nCr
\providecommand{\nPr}[2]{\,^{#1}P_{#2}} % nPr
\providecommand{\mbf}{\mathbf}
\providecommand{\pr}[1]{\ensuremath{\Pr\left(#1\right)}}
\providecommand{\qfunc}[1]{\ensuremath{Q\left(#1\right)}}
\providecommand{\sbrak}[1]{\ensuremath{{}\left[#1\right]}}
\providecommand{\lsbrak}[1]{\ensuremath{{}\left[#1\right.}}
\providecommand{\rsbrak}[1]{\ensuremath{{}\left.#1\right]}}
\providecommand{\brak}[1]{\ensuremath{\left(#1\right)}}
\providecommand{\lbrak}[1]{\ensuremath{\left(#1\right.}}
\providecommand{\rbrak}[1]{\ensuremath{\left.#1\right)}}
\providecommand{\cbrak}[1]{\ensuremath{\left\{#1\right\}}}
\providecommand{\lcbrak}[1]{\ensuremath{\left\{#1\right.}}
\providecommand{\rcbrak}[1]{\ensuremath{\left.#1\right\}}}
\theoremstyle{remark}
\newtheorem{rem}{Remark}
\newcommand{\sgn}{\mathop{\mathrm{sgn}}}
\providecommand{\abs}[1]{\left\vert#1\right\vert}
\providecommand{\res}[1]{\Res\displaylimits_{#1}} 
\providecommand{\norm}[1]{\lVert#1\rVert}
\providecommand{\mtx}[1]{\mathbf{#1}}
\providecommand{\mean}[1]{E\left[ #1 \right]}
\providecommand{\fourier}{\overset{\mathcal{F}}{ \rightleftharpoons}}
\providecommand{\ztrans}{\overset{\mathcal{Z}}{ \rightleftharpoons}}

%\providecommand{\hilbert}{\overset{\mathcal{H}}{ \rightleftharpoons}}
\providecommand{\system}{\overset{\mathcal{H}}{ \longleftrightarrow}}
	%\newcommand{\solution}[2]{\textbf{Solution:}{#1}}
\newcommand{\solution}{\noindent \textbf{Solution: }}
\providecommand{\dec}[2]{\ensuremath{\overset{#1}{\underset{#2}{\gtrless}}}}
\numberwithin{equation}{section}
%\numberwithin{equation}{subsection}
%\numberwithin{problem}{subsection}
%\numberwithin{definition}{subsection}

%\renewcommand{\thefigure}{\theproblem.\arabic{figure}}
\renewcommand{\thefigure}{\arabic{section}.\arabic{figure}}
\makeatletter
\@addtoreset{figure}{section}
\makeatother

%\numberwithin{figure}{subsection}

\def\putbox#1#2#3{\makebox[0in][l]{\makebox[#1][l]{}\raisebox{\baselineskip}[0in][0in]{\raisebox{#2}[0in][0in]{#3}}}}
     \def\rightbox#1{\makebox[0in][r]{#1}}
     \def\centbox#1{\makebox[0in]{#1}}
     \def\topbox#1{\raisebox{-\baselineskip}[0in][0in]{#1}}
     \def\midbox#1{\raisebox{-0.5\baselineskip}[0in][0in]{#1}}

\vspace{3cm}

\title{ 
%\logo{
%}
Pingala Series
%	\logo{Octave for Math Computing }
}
%\title{
%	\logo{Matrix Analysis through Octave}{\begin{center}\includegraphics[scale=.24]{tlc}\end{center}}{}{HAMDSP}
%}


% paper title
% can use linebreaks \\ within to get better formatting as desired
%\title{Matrix Analysis through Octave}
%
%
% author names and IEEE memberships
% note positions of commas and nonbreaking spaces ( ~ ) LaTeX will not break
% a structure at a ~ so this keeps an author's name from being broken across
% two lines.
% use \thanks{} to gain access to the first footnote area
% a separate \thanks must be used for each paragraph as LaTeX2e's \thanks
% was not built to handle multiple paragraphs
%

\author{Gautam Singh}
% note the % following the last \IEEEmembership and also \thanks - 
% these prevent an unwanted space from occurring between the last author name
% and the end of the author line. i.e., if you had this:
% 
% \author{....lastname \thanks{...} \thanks{...} }
%                     ^------------^------------^----Do not want these spaces!
%
% a space would be appended to the last name and could cause every name on that
% line to be shifted left slightly. This is one of those "LaTeX things". For
% instance, "\textbf{A} \textbf{B}" will typeset as "A B" not "AB". To get
% "AB" then you have to do: "\textbf{A}\textbf{B}"
% \thanks is no different in this regard, so shield the last } of each \thanks
% that ends a line with a % and do not let a space in before the next \thanks.
% Spaces after \IEEEmembership other than the last one are OK (and needed) as
% you are supposed to have spaces between the names. For what it is worth,
% this is a minor point as most people would not even notice if the said evil
% space somehow managed to creep in.



% The paper headers
%\markboth{Journal of \LaTeX\ Class Files,~Vol.~6, No.~1, January~2007}%
%{Shell \MakeLowercase{\textit{et al.}}: Bare Demo of IEEEtran.cls for Journals}
% The only time the second header will appear is for the odd numbered pages
% after the title page when using the twoside option.
% 
% *** Note that you probably will NOT want to include the author's ***
% *** name in the headers of peer review papers.                   ***
% You can use \ifCLASSOPTIONpeerreview for conditional compilation here if
% you desire.




% If you want to put a publisher's ID mark on the page you can do it like
% this:
%\IEEEpubid{0000--0000/00\$00.00~\copyright~2007 IEEE}
% Remember, if you use this you must call \IEEEpubidadjcol in the second
% column for its text to clear the IEEEpubid mark.



% make the title area
\maketitle

%\newpage

\tableofcontents

\bigskip

\begin{abstract}
This manual provides a simple introduction to Transforms
\end{abstract}
\fi
\section{JEE 2019}
\begin{align}
	\label{eq:pingala/10-orig-diff-a}
	a_n &= \frac{\alpha^{n}-\beta^{n}}{\alpha - \beta}, \quad n \ge 1
	\\
	b_n &= a_{n-1} + a_{n+1}, \quad n \ge 2, \quad b_1 =1
	\label{eq:pingala/10-orig-diff}
\end{align}
where $\alpha$ and $\beta$ ($\alpha > \beta$) are the roots of the
\begin{align}
z^2 - z - 1 = 0
\end{align}
%
Verify the following using a python code.
\begin{enumerate}[label=\thesection.\arabic*,ref=\thesection.\theenumi]
\item 
	\label{itm:ping-1}
\begin{align}
	\label{eq:ping-1}
	\sum_{k=1}^{n}a_k = a_{n+2}-1, \quad n \ge 1
\end{align}
 \item 
	\label{itm:ping-2}
\begin{align}
	\label{eq:ping-2}
	\sum_{k=1}^{\infty}\frac{a_k}{10^k} =\frac{10}{89}
\end{align}
 \item 
	\label{itm:ping-3}
\begin{align}
	\label{eq:ping-3}
	b_n =\alpha^n + \beta^n, \quad n \ge 1
\end{align}
 \item 
	\label{itm:ping-4}
\begin{align}
	\label{eq:ping-4}
	\sum_{k=1}^{\infty}\frac{b_k}{10^k} =\frac{8}{89}
\end{align}
\solution
\begin{lstlisting}
$ python3 pingala/codes/1.py
\end{lstlisting}
\end{enumerate}
\section{Pingala Series}
\begin{enumerate}[label=\thesection.\arabic*,ref=\thesection.\theenumi]
	\item The {\em Pingala} series is generated using the difference equation 
\begin{align}
	x(n+2) = x\brak{n+1} + x\brak{n},  \quad x(0) = x(1) = 1, n \ge 0
	\label{eq:pingala/10-pingala}
\end{align}
Generate a stem plot for $x(n)$.
\\
\solution
The following code generates
    Fig. \ref{fig:pingala/xn}.
\begin{lstlisting}
$ python3 pingala/codes/2_1.py
\end{lstlisting}
\begin{figure}[!htp]
    \includegraphics[width=\columnwidth]{pingala/figs/2_1.png}
    \caption{Plot of $x(n)$}
    \label{fig:pingala/xn}
\end{figure}
\item The {\em one sided} $Z$-transform of $x(n)$ is defined as 
\begin{align}
	X^{+}(z) = \sum_{n = 0}^{\infty}x(n)z^{-n}, \quad z \in \mathbb{C}
\label{eq:pingala/one-Z}
\end{align}
Find $X^{+}(z)$.
\\
\solution Taking the one-sided $Z$-transform on both sides of \eqref{eq:pingala/10-pingala},
\begin{align}
	\mathcal{Z}^+\sbrak{x(n + 2)} &= \mathcal{Z}^+\sbrak{x(n + 1)} + \mathcal{Z}^+\sbrak{x(n)} \\
\implies     z^2X^+(z) - z^2x(0) - zx(1) &= zX^+(z) - zx(0) + zX^+(z) \\
 \implies   \brak{z^2 - z - 1}X^+(z) &= z^2 \\
  \implies    X^+(z) = \frac{1}{1 - z^{-1} - z^{-2}} 
    &= \frac{1}{\brak{1 - \alpha z^{-1}}\brak{1 - \beta z^{-1}}}, \quad |z| > \alpha
    \label{eq:pingala/X-z}
\end{align}
\item Find $x(n)$.
\\
\solution Expanding $X^+(z)$ in \eqref{eq:pingala/X-z} using partial fractions, we get
\begin{align}
    X^+(z) &= \frac{1}{\brak{\alpha - \beta}}\sbrak{\frac{z}{1 - \alpha z^{-1}} - \frac{z}{1 - \beta z^{-1}}} \\
	\implies    x(n) &= \frac{\alpha^{n + 1} - \beta^{n + 1}}{\alpha - \beta}u(n) 
	\\
	&= a_{n + 1}
    \label{eq:pingala/x-n-def}
\end{align}
upon comparing with
	\eqref{eq:pingala/10-orig-diff-a}.
\end{enumerate}
\section{Linear Time Invariant System}
\begin{enumerate}[label=\thesection.\arabic*,ref=\thesection.\theenumi]
	\item Sketch 
\begin{align}
	y(n) = x\brak{n-1} + x\brak{n+1},  \quad n \ge 0
	\label{eq:pingala/10-orig-diff-rev}
\end{align}
\solution
Execute
\begin{lstlisting}
$ python3 pingala/codes/2_2.py
\end{lstlisting}
to obtain Fig. 
    \ref{fig:pingala/yn}
\begin{figure}[!htbp]
    \includegraphics[width=\columnwidth]{pingala/figs/2_2.png}
    \caption{Plot of $y(n)$}
    \label{fig:pingala/yn}
\end{figure}
\item Show that 
\begin{align}
	x(n + 1)&\system{Z}  zX^+(z) - zx(0)
	\\
	x(n - 1)&\system{Z}   z^{-1}X^+(z) + zx(-1) 
\end{align}
\item Find $Y^{+}(z)$. 
	\\
\solution Taking the one-sided $Z$-transform on both sides of \eqref{eq:pingala/10-orig-diff-rev},
\begin{align}
	\mathcal{Z}^+\sbrak{y(n)} &= \mathcal{Z}^+\sbrak{x(n + 1)} + \mathcal{Z}^+\sbrak{x(n - 1)} \\
	Y^+(z) &= zX^+(z) - zx(0) + z^{-1}X^+(z) + zx(-1) \\
&= \frac{z + z^{-1}}{1 - z^{-1} - z^{-2}} - z 
= \frac{1 + 2z^{-1}}{1 - z^{-1} - z^{-2}}, \quad |z| > \alpha
\end{align}
\item Show that
\begin{align}
	\label{eq:itm-3}
	y(n) = b_{n + 1}.
\end{align}
\item Find the impulse response of 
	\eqref{eq:pingala/10-orig-diff-rev}
\iffalse
\item Find $y(n)$.
    \label{pr:1-3}
    \\
\solution 
Using \eqref{eq:pingala/X-z},
\begin{align}
	Y^+(z) &= \brak{1 + 2z^{-1}}\sum_{n = 0}^{\infty}x(n)z^{-n} \\
           &= \sum_{n = 0}^{\infty}x(n)z^{-n} + \sum_{n = 1}^{\infty}2x(n - 1)z^{-n} \\
           &= x(0) + \sum_{n = 1}^{\infty}\brak{x(n) + 2x(n - 1)}z^{-n}
	   \\
           &= y(0) + \sum_{n = 1}^{\infty}y(n)z^{-n}
\end{align}
using the definition of $Y^{+}(z)$.
Thus, 
\begin{align}
	y(n) =
	\begin{cases}
		x(n) & n \le 0
		\\
		x(n) + 2x(n - 1) & n \ge 1
	\end{cases}
\end{align}
Substituting from 
    \eqref{eq:pingala/x-n-def},
\begin{align}
    y(n) &= \frac{\brak{\alpha^{n + 1} - \beta^{n + 1}} + \brak{2\alpha^n + 2\beta^n}}{\alpha - \beta} \\
         &= \frac{\brak{\alpha^{n + 2} - \beta^{n + 2}} + \brak{\alpha^{n} + \beta^{n}}}{\alpha - \beta}  \\
         &= \frac{\brak{\alpha^{n + 2} - \beta^{n + 2}} - \alpha\beta\brak{\alpha^{n} + \beta^{n}}}{\alpha - \beta} \\
         &= \frac{\brak{\alpha - \beta}\brak{\alpha^{n + 1} + \beta^{n + 1}}}{\alpha - \beta} \\
         &= \alpha^{n + 1} + \beta^{n + 1}, \quad  n \ge 0
\label{eq:pingala/y-b}
\end{align}
 $\because \alpha + \beta = 1$.
 From 
	\eqref{eq:pingala/10-orig-diff},
    \eqref{eq:pingala/x-n-def}
    and
	\eqref{eq:pingala/10-orig-diff-rev},
%Comparing \eqref{eq:pingala/y-b} with the definition of $b_n$, we see that 
Hence,
\begin{align}
	y(n) &= b_{n + 1}.
	\\
	\implies  b_n &= \alpha^n + \beta^n, \quad n \ge 1
\end{align}
and option 
	\ref{eq:ping-3}
	is correct.
\fi
\end{enumerate}
\section{Power of the Z transform}
\begin{enumerate}[label=\thesection.\arabic*,ref=\thesection.\theenumi]
\item Show that 
\begin{align}
	\sum_{k=1}^{\infty}\frac{a_k}{10^k}= 
	\frac{1}{10}\sum_{k=0}^{\infty}\frac{x\brak{k}}{10^k} =\frac{1}{10}X^{+}\brak{{10}}
\end{align}
\label{pr:1-2}
\solution 
\begin{align}
    \sum_{k=1}^{\infty}\frac{a_k}{10^k} &= \frac{1}{10}\sum_{k = 0}^{\infty}\frac{a_{k+1}}{10^k} 
                                        = \frac{1}{10}\sum_{k = 0}^{\infty}\frac{x(k)}{10^k} \\
                                        &= \frac{1}{10}X^+(10) 
                                        = \frac{1}{10}\times\frac{100}{89} = \frac{10}{89}
\end{align}
Thus,
\eqref{eq:ping-2} is correct.
 \item Show that 
\begin{align}
	\sum_{k=1}^{\infty}\frac{b_k}{10^k} =
	\frac{1}{10}\sum_{k=0}^{\infty}\frac{y\brak{k}}{10^k} =\frac{1}{10}Y^{+}\brak{{10}}
\end{align}
\label{pr:1-4}
\solution
\begin{align}
    \sum_{k=1}^{\infty}\frac{b_k}{10^k} &= \frac{1}{10}\sum_{k = 0}^{\infty}\frac{b_{k+1}}{10^k} 
                                        = \frac{1}{10}\sum_{k = 0}^{\infty}\frac{y(k)}{10^k} \\
                                        &= \frac{1}{10}Y^+(z) 
                                        = \frac{1}{10}\times\frac{120}{89} = \frac{12}{89}
\end{align}
Thus,
\eqref{eq:ping-4} is incorrect.
\item Show that 
\begin{align}
	\alpha^n + \beta^n, \quad n \ge 1
    \label{eq:pingala/yn-exp}
\end{align}
can be expressed as 
\begin{align}
	w(n) = \brak{\alpha^{n+1} + \beta^{n+1}}u(n)
\end{align}
and find $W(z)$.
\\
\solution Putting $n = k + 1$ in \eqref{eq:pingala/yn-exp} and using the definition of $u(n)$, 
\begin{align}
\alpha^n + \beta^n = \brak{\alpha^{k + 1} + \beta^{k + 1}}u(k)
\end{align}
Hence, \eqref{eq:pingala/yn-exp} can be expressed as
\begin{align}
w(n) = \brak{\alpha^{n+1} + \beta^{n+1}}u(n) 
\end{align}
Therefore,
\begin{align}
    W(z) = Y(z) = \frac{1 + 2z^{-1}}{1 - z^{-1} - z^{-2}}
\end{align}
Thus, by invoking 
	\eqref{eq:itm-3},
we find that 
\eqref{eq:ping-3}
is correct 
\end{enumerate}
\section{Convolution}
\begin{enumerate}[label=\thesection.\arabic*,ref=\thesection.\theenumi]
\item Show that 
\begin{align}
	\sum_{k=1}^{n}a_k = 
	\sum_{k=0}^{n-1}x(k) = x(n)*u(n-1)
\end{align}
\solution From \eqref{eq:pingala/x-n-def}, and noting that $x(n) = 0\ \forall\ n < 0$,
\begin{align}
    \sum_{k=1}^{n}a_k &= \sum_{k=0}^{n-1}x(k) 
                      = \sum_{k = -\infty}^{n - 1}x(k) \\
                      &= \sum_{k = -\infty}^{\infty}x(k)u(n - 1 - k) 
                      = x(n)*u(n - 1)
\end{align}
\item Show that 
\begin{align}
a_{n+2}-1, \quad n \ge 1
\end{align}
can be expressed as 
\begin{align}
	\sbrak{x\brak{n+1}-1}u\brak{n-1}
\end{align}
\solution From \eqref{eq:pingala/x-n-def},
\begin{align}
    a_{n+2} - 1 = \sbrak{x(n + 1) - 1}, \quad n \ge 1
\end{align}
and so, using the definition of $u(n)$,
\begin{align}
    a_{n+2} - 1 = \sbrak{x(n + 1) - 1}u(n-1)
\end{align}
\item Show that
\begin{align}
	\sbrak{x(n + 1) - 1}u(n-1)&\system{Z} 
     \frac{z^{-1}}{\brak{1 - z^{-1} - z^{-2}}\brak{1 - z^{-1}}} 
\end{align}
\solution The Z transform of the above signal can be expressed as
\begin{align}
	\sum_{n = 1}^{\infty}x(n + 1) z^{-n} -\frac{z^{-1}}{1-z^{-1}}
	&=\sum_{n = 2}^{\infty}x(n) z^{-n+1} -\frac{z^{-1}}{1-z^{-1}}
	\\
	&=z\sbrak{X^{+}(z) - x(0) -x(1)z^{-1}} -\frac{z^{-1}}{1-z^{-1}}
	\\
	&= \frac{z}{1 - z^{-1} - z^{-2}} - z - 1 - \frac{z^{-1}}{1 - z^{-1}} \\
	&= \frac{z}{1 - z^{-1} - z^{-2}} -  \frac{z}{1 - z^{-1}} \\
	&=\frac{z^{-1}}{\brak{1 - z^{-1} - z^{-2}}\brak{1 - z^{-1}}} 
\end{align}
From \eqref{eq:pingala/x-n-def}, we get
\begin{align}
    \sum_{k = 1}^{n}a_k = a_{n+2} - 1
\end{align}
\end{enumerate}

\iffalse
\section{Problem}
\begin{frame}
\frametitle{Problem Statement}
%
Balance the following chemical equation
\begin{align}
\label{eq:chem_balance}
Fe+H_2O &\rightarrow Fe_3O_4 + H_2
\end{align}
%
using a matrix equation.
%A circle $C$ passes through 
%\begin{equation} 
%\vec{P}=\myvec{-2\\ 4} 
%\label{eq:circle_7_p}
%\end{equation} 
%and touches the $y$-axis at 
%\begin{equation} 
%\vec{Q}=\myvec{0\\ 2}. 
%\label{eq:circle_7_q}
%\end{equation}
%Which one of the  following equations can represent a diameter of this circle?
%\begin{enumerate}[label=(\roman*)]
%\begin{multicols}{2}
%\setlength\itemsep{1em}
%\item $\myvec{4 & 5}\vec{x} = 6 $
%\item $\myvec{2 & -3}\vec{x} +10 = 0 $
%\item $\myvec{3 & 4}\vec{x} = 3 $
%\item $\myvec{5 & 2}\vec{x} +4= 0 $
%\end{multicols}
%\end{enumerate}
\end{frame}

%\subsection{Literature}
\section{Solution}
\subsection{Linear Equation}
\begin{frame}
\frametitle{Linear Equation}
%\framesubtitle{Literature}
Let the balanced version of \eqref{eq:chem_balance} be 
%
\begin{align}
\label{eq:chem_balance_unsol}
x_1Fe+x_2H_2 O &\rightarrow x_3Fe_3 O_4 + x_4H_2
\end{align}
%
which results in the following equations
%
\begin{align}
\begin{split}
\brak{x_1 -3x_3}Fe &= 0
\\
\brak{2x_2 -2x_4}H &= 0
\\
\brak{x_2 -4x_3}H &= 0
\end{split}
\end{align}
%Let $\vec{O}$ be the centre of $C$. Then the equation of the normal, OQ is
%\begin{align}
%%\vec{x}^T\vec{x}-2\vec{O}^T\vec{x} +F = 0
%\myvec{0 & 1}\brak{\vec{O}-\vec{Q}} &= 0
%\nonumber \\ 
%\implies \myvec{0 & 1}\vec{O} = 2
%\label{eq:circle_7_o1}
%\end{align}
%%
%Also, 
%%Substituting \eqref{eq:circle_7_p} in \eqref{eq:circle_7_c}, 
%\begin{align}
%\norm{\vec{O}-\vec{P}}^2&=\norm{\vec{O}-\vec{Q}}^2 
%\nonumber \\
%\implies 2\brak{\vec{P}-\vec{Q}}^T\vec{O} &= \norm{\vec{P}}^2-\norm{\vec{Q}}^2 
%\nonumber \\
%\text{or, } \myvec{1 & -1}\vec{O} &= -4
%\label{eq:circle_7_o2}
%\end{align}
%%
%\eqref{eq:circle_7_o1} and \eqref{eq:circle_7_o2} result in the matrix equation
%\begin{align}
%\myvec{1 & -1 \\ 0 & 1}\vec{O} = \myvec{-4\\2}
%\label{eq:circle_7_matrix}
%\end{align}
%yielding the augmented matrix
%\begin{align}
%\myvec{1 & -1 & -4\\ 0 & 1 & 2} \leftrightarrow \myvec{1 & 0 & -2\\ 0 & 1 & 2}\implies \vec{O} = \myvec{-2 \\2}
%\label{eq:circle_7_o}
%\end{align}
%%
\end{frame}
\subsection{Matrix Equation}
\begin{frame}
\frametitle{Matrix Equation}
The linear equations can be expxressed as
%\begin{enumerate}[label=(\roman*)]
\begin{align}
\begin{split}
x_1 + 0.x_2 -3x_3 +0.x_4&= 0
\\
0.x_1+2x_2 +0.x_3-2x_4 &= 0
\\
0.x_1+x_2 -4x_3+ 0.x_4 &= 0
\end{split}
\end{align}
%
resulting in the matrix equation
\begin{align}
\label{eq:chem_balance_mat_eq}
\begin{split}
\myvec{
1 & 0 & -3 & 0
\\
0 & 2 & 0 & -2
\\
0 & 1 & -4 & 0
}
\vec{x} &= \vec{0}
\end{split}
\end{align}
where
\begin{align}
\vec{x} = \myvec{x_1 \\ x_2 \\ x_3 \\ x_4} 
\end{align}
%\item $\myvec{4 & 5}\vec{O} = 2 \ne 6 $. Incorrect.
%\vfill
%\item $\myvec{2 & -3}\vec{O} +10 = 0 $. Correct.
%\vfill
%\item $\myvec{3 & 4}\vec{O} = 2 \ne 3 $.  Incorrect.
%\vfill
%\item $\myvec{5 & 2}\vec{O} +4= -2 \ne 0 $. Incorrect
%\end{enumerate}
\end{frame}
\subsection{Row Reduction}
\begin{frame}
\frametitle{Row Reduction}
\eqref{eq:chem_balance_mat_eq} can be row reduced as follows
%
\begin{align}
\label{eq:chem_balance_mat_row}
\myvec{
1 & 0 & -3 & 0
\\
0 & 2 & 0 & -2
\\
0 & 1 & -4 & 0
}
 \xleftrightarrow[]{R_2 \leftarrow \frac{R_2}{2}}
\myvec{
1 & 0 & -3 & 0
\\
0 & 1 & 0 & -1
\\
0 & 1 & -4 & 0
}
\\
 \xleftrightarrow[]{R_3\leftarrow R_3-R_2}
\myvec{
1 & 0 & -3 & 0
\\
0 & 1 & 0 & -1
\\
0 & 0 & -4 & 1
}
\\
 \xleftrightarrow[]{R_1\leftarrow 4R_1-3R_3}
\myvec{
4 & 0 & 0 & -3
\\
0 & 1 & 0 & -1
\\
0 & 0 & -4 & 1
}
\\
 \xleftrightarrow[R_3 \leftarrow -\frac{1}{4}R_3]{R_1\leftarrow \frac{1}{4}}
\myvec{
1 & 0 & 0 & -\frac{3}{4}
\\
0 & 1 & 0 & -1
\\
0 & 0 & 1 & -\frac{1}{4}
}
\end{align}
%
%The radius  of $C$ is obtained as
%\begin{align}
%r = \norm{O-P} = 2
%\end{align}
\end{frame}
%\section{Plot}
\subsection{Balanced Equation}
\begin{frame}[fragile]
\frametitle{Balanced Equation}

Thus, 
\begin{align}
\label{eq:chem_balance_mat_sol}
x_1 &= \frac{3}{4}x_4, x_2 = x_4, x_3 = \frac{1}{4}x_4
\\
\implies \vec{x} &= x_4\myvec{\frac{3}{4} \\ 1 \\ \frac{1}{4} \\ 1}= \myvec{3 \\ 4 \\ 1 \\ 4}
\end{align}
%
upon substituting $x_4 = 4$.
%
\eqref{eq:chem_balance_unsol} then becomes
%
\begin{align}
\label{eq:chem_balance_final}
3Fe+4H_2 O &\rightarrow Fe_3 O_4 + 4H_2
\end{align}
The code in 
{\footnotesize
\begin{lstlisting}
https://github.com/gadepall/school/blob/master/training/chemistry/codes/chembal.py
\end{lstlisting}
}
verifies \eqref{eq:chem_balance_mat_sol}.
%plots Fig. \ref{fig:circle_diameter}.
%
%\begin{figure}
%\centering
%\includegraphics[width=0.6\columnwidth]{./figs/circle_diameter.eps}
%\caption{Circle $C$ and all lines (i)-(iv). (ii) is a diameter.}
%\label{fig:circle_diameter}
%\end{figure}
\end{frame}
\fi
%\begin{frame}
%\frametitle{Introduction}
%\framesubtitle{Literature}
%%\begin{figure}[t!]
%%    \centering
%%    \begin{subfigure}[t]{0.4\columnwidth}
%%        \centering
%%        \includegraphics[width=\columnwidth]{point_source}
%%        \caption{Single point source}
%%\label{fig3:subfig1}        
%%    \end{subfigure}%
%%    ~ 
%%    \begin{subfigure}[t]{0.4\columnwidth}
%%        \centering
%%        \includegraphics[width=\columnwidth]{pointNoPowerDist_new}
%%        \caption{SNR profile}
%%\label{fig3:subfig2}
%%    \end{subfigure}
%%  %  \caption{Average SNR for a BPP. $N=16$}
%%    \label{fig3}
%%  \end{figure}
%
%\end{frame}
%  
%
%
%%

\end{document}
