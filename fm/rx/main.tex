\begin{enumerate}[label=\arabic*.,ref=\thesection.\theenumi]
\numberwithin{equation}{enumi}
\item List the various components used to implement receiver.
\\
\solution
\\
\begin{table}[!ht]
  \centering
  \input{fm/rx/table.tex}
  \caption{Components Required}
  \label{tab:components}
\end{table}
Components are listed in the table \tabref{tab:components}\\
The picture of RTL-SDR and Antenna is given in \figref{fig:rtl-sdr}.This set is used to receive the Fm signals.
\begin{figure}[H]
\centering
\includegraphics[width=0.5\columnwidth]{fm/rx/figs/rtl-sdr.png}
\caption{RTL-SDR}
\label{fig:rtl-sdr}
\end{figure}
\item Install and open GNU Radio using the following commands
\\
\begin{lstlisting}
sudo apt update
sudo apt install gnuradio
gnuradio-companion
\end{lstlisting}
\item How to construct the block diagram in GNU radio? \\
	\solution  \\
\textbf{Step-1}:\\
Search for Low pass filter block and add drag it to the work space.
\begin{figure}[H]
\centering
\includegraphics[width=\columnwidth]{fm/rx/figs/add.png}
\caption{Adding blocks}
\label{fig:add blocks}
\end{figure}
Similarly do for RTL-Source block,WBFM and Audio sink blocks.
\textbf{Step-2}:
Connect them according to the flowgraph shown in \figref{fig:Block_diagram}.
\begin{figure}[H]
\centering
\includegraphics[width=\columnwidth]{fm/rx/figs/block_diagram.png}
\caption{Block diagram of GNU Radio}
\label{fig:Block_diagram}
\end{figure}
\textbf{Note}:
Refer the following website for any queries.
\begin{lstlisting}
https://wiki.gnuradio.org/index.php?title=Creating_Your_First_Block
\end{lstlisting}
\item Explain each block in block diagram \\
	\solution \\
\textbf{1. RTL-SDR Source}:\\
The RTL-SDR Source block is used to stream samples from RTL-SDR device.
\begin{figure}[H]
\centering
\includegraphics[width=0.4\columnwidth]{fm/rx/figs/source_block.png}
\caption{RTL-SDR Source Block}
\label{fig:source block}
\end{figure}
\textbf{2. Low Pass Filter}:\\
Low Pass Filter removes frequencies that are higher than the cutoff.
\begin{figure}[H]
\centering
\includegraphics[width=0.3\columnwidth]{fm/rx/figs/lpf_block.png}
\caption{Low Pass Filter Block}
\label{fig:lpf}
\end{figure}
\textbf{3. WBFM}:\\
WBFM Receive is a block for demodulating a broadcast FM signal. The output is the demodulated audio.
\begin{figure}[H]
\centering
\includegraphics[width=0.3\columnwidth]{fm/rx/figs/wbfm_block.png}
\caption{WBFM Block}
\label{fig:wbfm block}
\end{figure}
\textbf{4. Audio sink}:\\
Audio Sink block provides audio output signal at the computer speakers.
\begin{figure}[H]
\centering
\includegraphics[width=0.3\columnwidth]{fm/rx/figs/audio_sink.png}
\caption{Audio Sink Block}
\label{fig:audio sink block}
\end{figure}
Connect the RTL-SDR to the system and execute the flowgraph in \figref{fig:Block_diagram}.\\
We can listen to the sound which we have transmitted.\\
A default python code is generated and stored as
\begin{lstlisting}
fm/rx/codes/fm_receiver.py
\end{lstlisting}
\item How to replace LPF default code with custom code?\\
\solution \\
\textbf{Step-1}:
Disable Low pass filter block and search for the python block and add it to the workspace.
\begin{figure}[H]
\centering
\includegraphics[width=\columnwidth]{fm/rx/figs/step_1.png}
\caption{search for blocks}
\label{fig:search for blocks}
\end{figure}
\textbf{Step-2}:
Double-click the block to edit the properties as in \figref{fig:Edit properties}
\begin{figure}[H]
\centering
\includegraphics[width=0.7\columnwidth]{fm/rx/figs/step_2.png}
\caption{Edit properties}
\label{fig:Edit properties}
\end{figure}
\textbf{Step-3}:
Change the block name as required by editing in the code as shown in \figref{fig:changing block name}                                               
\begin{figure}[H]
\centering
\includegraphics[width=\columnwidth]{fm/rx/figs/step_3.png}
\caption{Changing block name}
\label{fig:changing block name}
\end{figure}
\textbf{Step-4}:
Change the parameter name as required by editing in the code as shown in \figref{fig:changing parameter name}                                               
\begin{figure}[H]
\centering
\includegraphics[width=0.7\columnwidth]{fm/rx/figs/step_4.png}
\caption{Changing parameter name}
\label{fig:changing parameter name}
\end{figure}
Replace the below code with default code
\begin{lstlisting}
fm/rx/codes/lpf_block.py
\end{lstlisting} 
\item Read the data from RTL SDR using python code and without using GNU radio.
\\
\solution \\
The following code reads the data from RTL-SDR without GNU radio.
\begin{lstlisting}
fm/rx/codes/source_own.py
\end{lstlisting}
\item Design Low pass filter block without using GNU Radio
\\
\solution \\
The following code works as low pass filter.
\begin{lstlisting}
fm/rx/codes/lpf_own.py
\end{lstlisting}
\end{enumerate}