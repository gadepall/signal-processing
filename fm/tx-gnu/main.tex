\begin{enumerate}[label=\arabic*.,ref=\thesection.\theenumi]
\numberwithin{equation}{enumi}
\item List the various components used to implement transmitter.
\\
\solution
\\
\begin{table}[!ht]
  \centering
  \input{fm/tx-gnu/table.tex}
  \caption{Components Required}
  \label{tab:txcomponents}
\end{table}
Components are listed in the table \tabref{tab:txcomponents}\\
The picture of USRP and Antenna is given in \figref{fig:USRP}.This set is used to transmit the FM signals.
\begin{figure}[H]
\centering
\includegraphics[width=0.5\columnwidth]{fm/tx-gnu/figs/USRP.png}
\caption{USRP}
\label{fig:USRP}
\end{figure}
\item Install and open GNU Radio using the following commands
\\
\begin{lstlisting}
sudo apt update
sudo apt install gnuradio
gnuradio-companion
\end{lstlisting}
\item How to construct the block diagram in GNU radio? \\
	\solution  \\
\textbf{Step-1}:\\
Search for Wav file source block and drag it to the work space.
\begin{figure}[H]
\centering
\includegraphics[width=\columnwidth]{fm/tx-gnu/figs/wav_file.png}
\caption{Adding blocks}
\label{fig:add blocks}
\end{figure}
\textbf{Step-2}:
Similarly do for WBFM transmit,Rational resampler and UHD:USRP sink block.
\\
\textbf{Step-3}:
Change the parameters in each block according to your values by double cliking on it.
\\
\textbf{Step-4}:
Connect them according to the flowgraph shown in \figref{fig:Tx_Block_diagram}.
\begin{figure}[H]
\centering
\includegraphics[width=\columnwidth]{fm/tx-gnu/figs/tx-blockdiagram.png}
\caption{Block diagram of Transmitter in GNU Radio}
\label{fig:Tx_Block_diagram}
\end{figure}
\item Explain each block in block diagram of Transmitter. \\
	\solution \\
\textbf{1. Wav file source block}:\\
The "WAV File Source" block in GNU Radio is used to read audio data from a WAV file and provide it as a continuous stream of samples with specified sample rate in GNU Radio.
\\
It allows you to use pre-recorded audio files as a source of input for your GNU Radio flowgraph.
\begin{figure}[H]
\centering
\includegraphics[width=0.4\columnwidth]{fm/tx-gnu/figs/wavfile.jpg}
\caption{Wav file Source Block}
\label{fig:Wav file source block}
\end{figure}
\textbf{2.WBFM Transmitter}:\\
The "WBFM Transmit" block in GNU Radio is used to transmit a wideband FM (WBFM) signal over a specified frequency range.
\\
It takes an audio stream as input and modulates it onto an RF carrier frequency using WBFM modulation techniques. 
\\
This block allows you to create and transmit FM 
signals in GNU Radio
\begin{figure}[H]
\centering
\includegraphics[width=0.3\columnwidth]{fm/tx-gnu/figs/wbfm-tx.jpg}
\caption{WBFM Transmit Block}
\label{fig:wbfm_tx}
\end{figure}
\textbf{3.Rational Resampler}:\\
The "Rational Resampler" block in GNU Radio is used for resampling a signal by a rational factor.
\\
Resampling is the process of changing the sample rate of a signal while preserving its content and characteristics.
\\
The main function of the "Rational Resampler" block is to change the sample rate of the input signal according to the specified rational 
resampling factor.
\begin{figure}[H]
\centering
\includegraphics[width=0.3\columnwidth]{fm/tx-gnu/figs/resampler.jpg}
\caption{Rational resampler Block}
\label{fig:Rational resampler block}
\end{figure}
\textbf{4.USRP Sink}:\\
The "UHD: USRP Sink" block in GNU Radio is used to send data from a GNU Radio flowgraph to a Universal Software Radio Peripheral (USRP) device for transmission.
\\
It acts as an interface between the GNU Radio software and the USRP hardware
\begin{figure}[H]
\centering
\includegraphics[width=0.3\columnwidth]{fm/tx-gnu/figs/usrp-sink.jpg}
\caption{USRP Sink Block}
\label{fig:audio sink block}
\end{figure}
Connect USRP to the system and execute the flowgraph in \figref{fig:Tx_Block_diagram}.\\
Now we can transmit the FM signal through USRP.\\
A default python code is generated and stored as
\begin{lstlisting}
fm/tx-gnu/codes/fm_transmitter.py
\end{lstlisting}
\end{enumerate}