Play the message signal using
\begin{lstlisting}
sudo apt install ffmpeg
ffplay fm/input-audio/Sound.wav
\end{lstlisting}
\begin{enumerate}[label=\arabic*.,ref=\thesection.\theenumi]
\numberwithin{equation}{enumi}
\item Find the sampling rate of the message.
\\
	\solution
Executing	
\begin{lstlisting}
python3 fm/msg/codes/sample_rate.py
\end{lstlisting}
gives
the sampling rate of the input signal as 44100Hz.
\item Plot the spectrum of the message signal using the builtin FFT algorithm.\\
	\solution
The folowing code plots the spectrum in \figref{fig:FFTb} using the builtin FFT algorithm of the python library 'Numpy.'
\\
Executing
\begin{lstlisting}
python3 fm/msg/codes/msg_spec.py
\end{lstlisting}		
\begin{figure}[H]
\centering
\includegraphics[width=\columnwidth]{fm/msg/figs/msg_spec.pdf}
\caption{Plot of spectrum of message signal using builtin FFT algorithm.}
\label{fig:FFTb}
\end{figure}
\item Find the number of samples used to compute the FFT.\\
	\solution
The following code finds the number of samples used to compute the FFT
\begin{lstlisting}
python3 fm/msg/codes/no_of_samples.py
\end{lstlisting}
and gives number of samples as 1226536
\item What does the following command do?
\begin{lstlisting}
f_i = np.fft.fftfreq(len(audio_data), d=1/sample_rate)
\end{lstlisting}
\solution
The np.fft.fftfreq function calculates the frequencies corresponding to the discrete Fourier transform (DFT) output by using the formula:\\
\begin{align}
\centering
f = \frac{k}{nd}
\end{align} 
for k=100,  
\begin{align*}
\centering
f = \frac{100}{1226536(1/44100)}
\end{align*}
\begin{align*}
\centering
f = 3.595 Hz
\end{align*}
%\begin{tabular}{|c|l|c|}
 %   \hline 
  %  \textbf{Parameter} & \textbf{Value} &\textbf{Description} \\ \hline
   % f  & varies with k  & Frequency \\
    %k  &  $0 \leq k \leq n-1$  &index of the DFT output component\\
    %n  & 1226536  &Length of the input signal\\  
    %d  & 1/44100 sec  & Sample space\\ \hline 
%\end{tabular}
 %   \\ \\


The np.fft.fftfreq function generates an array of length n, where each element represents the frequency corresponding to the DFT output component at the respective index.The index k ranges from 0 to n-1, and each index corresponds to a specific frequency component.\\
\begin{table}
  \centering
  \input{fm/msg/table.tex}
  \caption{Parameters of Message signal}
  \label{4.1}
\end{table}
\item Plot the spectrum of the message signal by writing your own FFT algorithm.\\
	\solution
\begin{figure}[H]
\centering
\includegraphics[width=\columnwidth]{fm/msg/figs/FFTalgorithm.pdf}
\caption{Plot of spectrum of message signal using own FFT algorithm.}
\label{fig:FFTo}
\end{figure}
The folowing code plots the spectrum in \figref{fig:FFTo} using the DFT defined in  \eqref{eq:app-dft-def}.
\begin{lstlisting}
python3 fm/msg/codes/FFTalgorithm.py
\end{lstlisting}
\item Compute and plot the PSD of the message signal using 
\eqref{eq:app-psd-def}.
\\
	\solution
Executing	
\begin{lstlisting}
python3 fm/msg/codes/msg_psd.py
\end{lstlisting}

\begin{figure}[H]
\centering
\includegraphics[width=\columnwidth]{fm/msg/figs/msg_psd.pdf}
\caption{Plot of PSD of the message signal.}
\label{fig:msg_PSD}
\end{figure}

\item Find the bandwidth of the message signal. Through a plot, explain the principle used for computing the bandwidth.\\
\solution 
Executing
\begin{lstlisting}
python3 fm/msg/codes/msg_bw.py
\end{lstlisting}
gives the bandwidth of the message signal as 2366.62454 Hz
\\
The threshold of the message signal from the plot is 3.4965 $\times 10^{16}$
\\
The maximum and minimum frequencies are 1183.31227 Hz and -1183.31227 respectively.
\\
Therefore bandwidth is\\
\begin{align}
\centering
f_{max}-f_{min} = 1183.31227-(-1183.31227) = 2366.62454  Hz
\end{align}
\iffalse
\begin{lstlisting}
/fm/FM/codes/input.py
\end{lstlisting}
\fi
\end{enumerate}
