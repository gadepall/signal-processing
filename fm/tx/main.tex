\begin{enumerate}[label=\arabic*.,ref=\thesection.\theenumi]
\numberwithin{equation}{enumi}
\item Construct a Transmitter circuit.
\\
\solution:
\begin{figure}[H]
\centering
  \begin{figure}[H]
\begin{circuitikz} [scale=.73,font=\footnotesize]

%\draw (5,5) -- (6.5,5) to[R=$R_4:47k\Omega$, label=$\mathrm{R_4}$, left] (6.5,2);

\draw (0,0) node[circ] {} to[C=$C_1$] node[pos=0.07,below left=1.5ex]{\SI{0.1}{\mu F}} ++(3.7,0) node[npn, anchor=B] (Q1) {$Q_1$};
\draw(4.85,-0.5)--(4.85,-4);
%\draw (0,0) -- (0,5) toR=$R_3 47k\Omega$;
\draw (0,0)--(0,5) to[R=$R_3$]  node[pos=0.05,below left=1.5ex]{\SI{47}{k\Omega}}(5,5);
\draw (3,0)--(3,2) to[R=$R_1$]  node[pos=0.05,below left=1.5ex]{\SI{1}{M\Omega}}(4.85,2)--(4.85,.5);
\draw (4.8,2)--(5,2) to[C=$C_2$]node[pos=1.8,below left=1.2ex]{\SI{0.1}{\mu F}} (6.5,2);
%\draw(4.85,2)--(4.85,2.5) toR=$R_2:10k\Omega$;
\draw (4.85,2)--(4.85,2.5) to[R=$R_2$]  node[pos=0.15,below left=1ex]{\SI{10}{k\Omega}}(4.85,5);
\draw (5,5) -- (6.5,5) to[ R]node[pos=-1.5,above left]{$R_4$} node[pos=-0.8,above left=1ex]{\SI{47}{k\Omega}} (6.5,2);
%\draw (5,5) -- (6.5,5) to[R=$R_4$]node[pos=0.1,above right=1ex]{\SI{47}{k\Omega}}(6.5,2);
\draw (6.5,2)--(6.5,-2) to[C=$C_3$] node[pos=-0.01,above left=1.5ex]{\SI{1}{ nF}} (6.5,-4);
\draw (0,0)--(0,-2) to [mic]node[pos=0.08,above right=2ex]{mic}(0,-4);
\draw(6.5,5)--(9,5);
\draw (9,5)--(9,4.5);

\draw(9,4.5)--(7.5,4.5)to[C=$C_6$]node[pos=0.01,above right=1.5ex]{\SI{33}{pF}} (7.5,2.5);
 %\draw(9,4.5)--(7.75,4.5)to[C]node[pos=-1.2,above left]{$C_6$} node[pos=-0.5,above left=1ex]{\SI{33}{pF}} (7.75,2.5);
\draw(9,4.5)to[C=$C_5$]node[pos=0.01,above right=1.5ex]{\SI{33}{pF}} (9,2.5);
\draw(7.5,4.5)--(10.5,4.5)to[L=$L_1$] node[pos=1.2,above right=2.5ex]{\SI{1}{\mu H}} (10.5,2.5);
\draw (7.5,2.5)--(10.5,2.5);
\draw(9,2.5)to[C=$C_4$] node[pos=0.01,below right=1.5ex]{\SI{10}{pF}} (9,-1.5);
\draw(9,-1.5)--(7.75,-1.5);
\draw(6.5,0)--(6.6,0);
\draw(0,0) ++(6.6,0) node[npn, anchor=B] (Q2){$Q_2$};
\draw(7.74,.5)--(7.74,1.5)--(9,1.5);
\draw(7.74,-0.5)--(7.74,-1.5);
\draw(7.74,-1.5)to[R=$R_5\ 470\Omega$] (7.74,-4);
\draw(9,1.5)--(9.75,1.5)to[C=$C_7$]node[pos=0.01,below right=1.5ex]{\SI{10}{pF}}  (10,1.5);
\draw(10,1.5)--(11,1.5);
\draw(11,1.5)--(11,2.5) node[antenna] {antenna};
\draw(9,5)--(12.5,5)to[C=$C_8$] node[pos=0.05,below left=1.5ex]{\SI{0.1}{\mu F}}(12.5,-4);
\draw(12.5,5)--(14.5,5);
\draw (14.5, -4) to [battery,l=12V](14.5, 5);
%\draw(14.5,-4)--(14.5,5)tobattery,l=\SI{12}{\V};
\draw(0,-4)--(14.5,-4);
\end{circuitikz}
%\captionof{figure}{FM Transmitter Circuit}
\label{fig:tx_cir}
\end{figure}

  \caption{FM Transmitter Circuit diagram}
  \label{ckt:circuit_2}
\end{figure}
\item The modulated signal is given by 
\begin{align}
	s(t) = \cos\brak{2\pi f_c t + \phi(t)}
\end{align}
where
\begin{align}
	\phi(t) = 2\pi k_f \int_{0}^{t}m(\tau)\,d\tau
	\label{4.2.1.2}
\end{align}
List the various parameters in a table.
\\
\solution
Parameters are in the table \tabref{tab:tab_2}
\vspace{10mm}
\begin{table}[!ht]
  \centering
  \input{fm/tx/table.tex}
  \caption{Parameters of signal}
  \label{tab:tab_2}
\end{table}

\item Obtain a difference equation for computing $\phi(t)$.  Suggest a sampling rate.
\\
\solution 
\quad To obtain the difference equation for computing $\phi(t)$, we need to discretize the integral in the given equation. We can use the rectangular rule for numerical integration.\\ 
Let us divide the interval $[t_0, t]$ into $N$ equal subintervals of width $\Delta t = (t - t_0)/N$. Then, we can approximate $m(\tau)$ by its value at the midpoint of each subinterval, $\tau_n = t_0 + (n + 1/2)\Delta t$, where $n = 0, 1, 2, ..., N-1$. This gives:
\begin{align*}
m(\tau_n) \approx m(n\Delta t)
\end{align*}
we can approximate the integral in the equation \ref{4.2.1.2}
\\
\begin{align*}
\phi(t) \approx 2\pi kf_c \Delta t \sum_{n=0}^{N-1} m(n\Delta t)
\end{align*}
$\phi$ at the $n$th time step $t_n = t_0 + n\Delta t$. Then, we can write:
\begin{equation}
\label{n}
\phi_n = 2\pi kf_c \Delta t \sum_{k=0}^{n-1} m(k\Delta t) \\
\end{equation}
\begin{equation}
\label{n-1}
\phi_{n-1} = 2\pi kf_c \Delta t \sum_{k=0}^{n-2} m(k\Delta t)
\end{equation}
Subtracting the \ref{n-1} from \ref{n}, we get:
\begin{equation}
\phi_n - \phi_{n-1} = 2\pi k_f f_c \Delta t, m((n-1)\Delta t)
\end{equation}
\item Plot the spectrum of the transmitted signal.
\\
\solution
The folowing code plots the spectrum of transmitted signal in \figref{fig:Trans}
\\
Executing
\begin{lstlisting}
python3 fm/tx/codes/tx_spec.py
\end{lstlisting}
\begin{figure}{H}
\centering	
\includegraphics[width=\columnwidth]{fm/tx/figs/tx_spec.pdf} 
\caption{Plot of spectrum of transmitted signal.}
\label{fig:Trans}
\end{figure}
\item Compute and plot the PSD of the message signal using 
\eqref{eq:app-psd-def}.
\\
	\solution
Executing	
\begin{lstlisting}
python3 fm/tx/codes/tx_psd.py
\end{lstlisting}

\begin{figure}[H]
\centering
\includegraphics[width=\columnwidth]{fm/tx/figs/tx_psd.pdf}
\caption{Plot of PSD of the message signal.}
\label{fig:fm_PSD}
\end{figure}
\item Compute the bandwidth of the transmitted signal.
\\
\solution
Executing
\begin{lstlisting}
python3 fm/tx/codes/tx_bw.py
\end{lstlisting}
gives the bandwidth of the transmitted signal as 7190.69574 Hz
\\
The threshold of the transmitted signal from the plot is 1.2621 $\times 10^{10}$
\\
The maximum and minimum frequencies are 3595.34787 Hz and -3595.34787 respectively.
\\
Therefore bandwidth is\\
\begin{align}
\centering
f_{max}-f_{min} = 3595.34787-(-3595.34787) = 7190.69574 Hz
\end{align}
\iffalse
\begin{lstlisting}
/fm/FM/codes/mod.py
\end{lstlisting}
\fi
\end{enumerate}
